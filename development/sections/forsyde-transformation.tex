\section{ForSyDe Transformation Approach}
\label{sec:forsyde-transformation-approach}

In this section, the general view of the ForSyDe transformation approach is explained. ForSyDe is a design methodology which its main goal is to raise the level of abstraction as a promising approach to overcome the problems of escalating complexity and rising heterogeneity of SoC and Cyber-Physical-Systems (CPSs) \cite{sander2017forsyde}. The functional programming paradigm and the theory of Model of Computations (MoCs) lay the foundations for the ForSyDe. In particular, the
higher abstraction level is achieved by exploiting the potential of the higher-order functions in functional programming, which provide the opportunity to abstract over processes \cite{sander2003system}.



\begin{figure}[htbp]
  \centering
  \scalebox{0.7}{\input{\pathFigures/forsyde-transformational-design-flow.pdf_t}}
  \caption{The ForSyDe design flow}
  \label{fig:forsyde-transformational-design-flow}
\end{figure}


\begin{figure}[htbp]
  \centering
  \input{\pathFigures/transformational-design-refinement.pdf_t}
  \caption{Transformational design refinement}
  \label{fig:transformational-design-refinement}
\end{figure}

\begin{figure}[htb]
  \centering
  \input{\pathFigures/design-transformation-step.pdf_t}
  \caption{Design transformation step}
  \label{fig:design-transformation step}
\end{figure}

\begin{figure}[htbp]
  \centering
  \input{\pathFigures/map-merge.pdf_t}
  \caption{Illustration of the transformation rule \textit{MapMerge}}
  \label{fig:map-merge}
\end{figure}

\begin{figure}[htbp]
  \centering
  \scalebox{0.8}{\input{\pathFigures/balanced-tree.pdf_t}}
  \caption{Illustration of the transformation rule $\mathit{BalancedTree}$}
  \label{fig:balanced-tree}
\end{figure}

\begin{figure}[htbp]
  \centering
  \scalebox{0.7}{\mbox{\input{\pathFigures/pipelined-tree.pdf_t}}}
  \caption{Illustration of the transformation rule \textit{PipelinedTree}}
  \label{fig:pipelined-tree}
\end{figure}

\begin{figure}[htbp]
  \centering
  \scalebox{0.7}{\mbox{\input{\pathFigures/balanced-pipelined-tree.pdf_t}}}
  \caption{Combining \textit{BalancedTree} and \textit{PipelinedTree}}
  \label{fig:balanced-pipelined-tree}
\end{figure}


%%% Local Variables:
%%% mode: latex
%%% TeX-master: "../paper"
%%% End:
